\documentclass[]{article}

\usepackage[a4paper, total={6in, 8in}]{geometry}
\usepackage{setspace, enumitem, multicol}
\usepackage{mathtools}
\usepackage{amssymb}
\usepackage{amsthm}

\DeclarePairedDelimiter\floor{\lfloor}{\rfloor}

\begin{document}

\section{Heaps}
Heaps are good for:
\begin{enumerate}[itemsep=1pt]
	\item Finding the min in $\Theta(1)$
	\item Finding the min with insertions
	\item Finding the min with insertions and deletions
	\item Finding the $k$th largest in $\Theta(k\log n)$
\end{enumerate}
\begin{center}
	\textbf{Heap Invariant: parent is $\leq$ all children}
\end{center}
Different implementations:
\begin{enumerate}
	\itemsep1pt
	\item balanced binary tree
	\item balanced binary tree backed by an array
	\item other trees/forests (e.g., binomial and Fibonacci heaps)
\end{enumerate}

\subsection{Binary Heap}
A binary heap is a balanced binary tree that fulfills the heap invariant. Balanced means that leaves may only differ in height by one.\\
Special case of binary heaps: add new leaves from left-to-right (in reading-order).\\
\\
Operations:
\begin{enumerate}[itemsep=1pt]
	\item Finding the min (easy -- it is the root!)
	\item Insert new elements
	\item Extract the min (delete/pop)
\end{enumerate}
\subsubsection{Insertion}
\begin{itemize}
	\itemsep1pt
	\item Add as last leaf (first empty on last level, left-to-right)
	\item Fix heap invariant moving up (until: at root or no fix)
\end{itemize}
One operation of insert takes at worst $O(\lg n)$ (word case: fix the branch from leaf insert position to the root).
\subsubsection{Extract min (delete)}
\begin{itemize}
	\itemsep1pt
	\item Replace root with last leaf (reading order).
	\item Fix heap invariant moving down (the opposite of insert.)
\end{itemize}
Same runtime as insert, since it is the same operation just backwards.
\subsection{Heap Sort}
\begin{multicols}{2}
\begin{enumerate}[itemsep=0pt,parsep=0pt]
	\item insert all elements into the heap
	\item while heap is not empty:
	\item \quad $x$ = extract-min from heap
	\item \quad emit $x$
\end{enumerate}
\columnbreak
\begin{itemize}[itemsep=0pt,parsep=0pt]
	\item[] Naive: $\Theta(n\log n)$
	\item[] $\Theta(n)$
	\item[] $\Theta(\log n)$
	\item[] $\;$
\end{itemize}
\end{multicols}
\noindent
Runtime: $\Theta(n\log n)$\\
Space: Naive $\Theta(n)$ -- can be done in $\Theta(1)$
\subsubsection{Tree Represented by Array}
\begin{itemize}[itemsep=0pt]
	\item node $A[i]$ is at array index $i$
	\item parent $i = \floor{\frac{i-1}{2}}$
	\item left child of $i = 2i + 1$
	\item right child of $i = 2i + 2$  
\end{itemize}
\noindent
This naturally defines a binary balanced tree:\\
add as last leaf $\Rightarrow$ place in first free slot in array\\
$O(1)$ time/space: we use the array in which input was given\\
\\
parent(i) = $\floor{\frac{i-1}{2}}$\\
Inserted node $i$ (value $A[i]$)\\
If parent is larger, fix heap invariant by pulling $A[i]$ up.\\
\\
pullup($i$):
\begin{enumerate}[itemsep=0pt,parsep=0pt]
	\item if $A[i] < A[parent(i)]$ // violation
	\item \quad exchange $A[i]$ with $A[parent(i)]$
	\item \quad pullup($parent(i)$)
\end{enumerate}
\noindent
analog: push down during extract min/delete\\\\

\textbf{loop invariant:} "heap invariant holds for subtree rooted in $i$"
\begin{itemize}[itemsep=0pt]
	\item initialization: a tree consisting of just $i$ is sorted/holds heap invariant
	\item maintenance: if $parent(i)$ violates heap invariant, swapping with $i$ solves it
	\item termination: loop ends when $i$ is root or when $parent(i)$ does not violate heap invariant
\end{itemize}
\noindent
Space complexity: $\Theta(1)$\\
Time complexity: $\Theta(\lg n)$\\
Inserting $n$ elements has runtime $\Theta(n \lg n)$
\subsubsection{Resizing the Array}
If the array becomes full, we resize by doubling the array. We can show that the runtime does not change because:\par
Cost for occasional doubling (across $n$ elements) is in $O(n)$; No change in runtime for inserting $n$ elements, which is $O(n \lg n)$ because $O(n \lg n + n) = O(n \lg n)$\\
\\
Let $c_i = \begin{cases}
	i & \text{if } i - 1 \text{ is a power of } 2\\
	1 & \text{otherwise }
\end{cases}
$\\
When power of two, we filled up the array and need to double.
\begin{align*}
	\sum_{i=1}^{n}c_i &\leq n + \sum_{j=0}^{\lg n}2^j\\
	&< n + 2n\\
	&< 3n
\end{align*}
So, total cost is upper bounded by $O(n)$
\subsubsection{Heap Construction}
Using the Tree Represented by Array, build heap bottom-up from unsorted list of $n$ elements.
\begin{enumerate}[itemsep=0pt,parsep=0pt]
	\item copy elements into array $A$
	\item for $i$ from $\frac{length}{2}-1$ down to $0$
	\item \quad pushdown($i$)
\end{enumerate}


\end{document}