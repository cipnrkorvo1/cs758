\documentclass[]{article}

\usepackage[a4paper, total={6in, 8in}]{geometry}
\usepackage{setspace}

%title: Sorting

\begin{document}

\subsection{Properties of Sort Algorithms}
\begin{itemize}
	\item \textbf{Sort keys with satellite data:} Given a list of (key,data) records, sort the pairs by key.
	\item \textbf{Stable sort:} Preserve order of multiple elements that have the same value
	\item \textbf{In-place sort:} No need to allocate a new array to write the output to; Sort using the input array
\end{itemize}

\noindent\fbox{%
	\parbox{\textwidth}{%
		\vspace{.8em}
		"in-place" is related to the question of space complexity. Regarding space complexity costs
		\begin{itemize}
			\item Space used for inputs comes for "free" (no cost).
			\item Book keeping space is counted.
			\item (If not in-place) space for the output is counted.
		\end{itemize}
	}%
}

\section{Counting Sort}
	\textit{TODO: 01/22/26 lecture}
	
	
\subsection{Stable Counting Sort}
	\begin{verbatim}
		count[k]: number of keys that fulfill (key == k)
		pos[k]: number of keys that fulfill (key < k)
		
		# sort
		for each (data, k) in inputs:
		    out[pos[k]] = (data, k)
		    pos[k]++
	\end{verbatim}
	Space complexity? $O(n+k)$ \hspace{1.3cm} Time complexity? $O(n+k)$
	
	
\section{Radix Sort}
	Idea: Sort keys digit-by-digit, start with \textbf{least-significant digit}.
	\begin{verbatim}
		# For n numbers with d digits (each digit has k values)
		for i from 0 to d:
		    stable sort on digit in place i from right
	\end{verbatim}
	\subsubsection*{space complexity?}
	The space complexity of the stable sort algorithm.\\
	For stable counting sort this is $O(n+k)$ since it is not an in-place algorithm.
	\subsubsection*{time complexity?}
	Run counting sort $d$ times.\\
	$\Theta(d(n+k))$\\
	for $n$ numbers with $d$ digits in range of $[1..k]$\\
	\\
	Considering $d$ a constant and assuming $k = O(n)$, radix sort runs in linear time!\\
	(beats $\Theta(n$ log $n))$ of comparison sort!!)

\section{Insertion Sort}
	Idea: consider input elements in order.\\
	Take the next element $A[i]$ and insert it at the right place among previous numbers.
	
	Space complexity? $O(n)$ \hspace{1.3cm} Time complexity? $O(n^2)$ 

\section{Merge Sort}
	'divide and conquer': divide, conquer, combine\\
	Idea:
	\vspace{-0.5em}
	\begin{enumerate}
		\itemsep0em 
		\item split input into two segments (divide),
		\item sort left segment by recursion (conquer),
		\item sort right segment by recursion,
		\item merge sorted segments (combine)
	\end{enumerate}
	
	\noindent Anchor: Any list with a single element is already sorted.\par
	Space complexity? $ $ \hspace{1.3cm} Time complexity? $O(n $ log $n)$
	

\section{Quicksort}

\section{Partition}

\section{Intro-Sort}


\end{document}