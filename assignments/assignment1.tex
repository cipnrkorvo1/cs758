\documentclass[]{article}

\usepackage[a4paper, total={6in, 8in}]{geometry}
\usepackage{amsmath, amsthm, setspace}

\theoremstyle{definition}
\newtheorem{definition}{Definition}[section]

\newtheorem{theorem}{Theorem}

\begin{document}
	
	\begin{center}
	\Large\bfseries Assignment 1
	\end{center}
	\begin{center}
		CS 758: Algorithms\\
		Grey Cole\\
		January 29, 2026
	\end{center}
	
	\section{Exercise 2.1–3 in CLRS}
	
	\begin{verbatim}
		for i from 0 until n:
		    if A[i] == v then return i
		return NIL
	\end{verbatim}
	
	\noindent \textbf{Invariant:} At the start of each loop at index $i$, value $v$ does not appear in the subarray $A[0..i-1]$.\\\\
	
	\noindent\textbf{1 \; Initialization.} Before the first iteration $(i = 0)$, $A[0..-1]$ is empty, so $v$ cannot be contained in it.\\
	\textbf{2 \; Induction.} If $A[i] = v$ the algorithm returns $i$ so the invariant no longer needs to hold. \\\noindent If $A[i] \neq v$ then $A[0..i-1] \cup A[i]$ will not contain $v$ so the invariant holds.\\
	\textbf{3 \; Termination.} If the loop finishes without returning (when $i = n$), the invariant tells us that $v$ cannot be contained within the input array $A$ (assuming $A = A[0..n-1]$), so the algorithm returns NIL instead.\\
	
	\section{Exercise 3.1–1 in CLRS}
	
	\begin{definition}[$\Theta$-notation]
		\label{thetadef}
		For a given function $g(n)$, we denote by $\Theta(g(n))$ the \textit{set of functions}:\\
		$\Theta(g(n)) = \{ f(n) :$ there exist positive constants $c_1$, $c_2$, and $n_0$ such that $0 \leq c_1g(n) \leq f(n) \leq c_2g(n)$ for all $n \geq n_0\}$.
	\end{definition}
	
	\begin{definition}[Asymptotically nonnegative]
		\label{nonneg}
		There is a value $n_0$ such that $f(n) > 0$ for all $n \geq n_0$. 
	\end{definition}
	
	\begin{proof}
		Because the max function and addition are both commutative, we can assume $f(n) \leq g(n)$ without loss of generality. Since both are asymptotically nonnegative, we can also assert $0 < f(n) \leq g(n)$ for all $n \geq n_0$. Using this inequality, we can simplify:
		\begin{align}
			\text{max}(f(n), g(n)) &= g(n)\\
			f(n) + g(n) &= a \cdot g(n) \text{ where } 1 \leq a \leq 2
		\end{align}
		We can also operate on $(2)$ to get the following:
		\begin{align}
			f(n) + g(n) - g(n) &= a \cdot g(n) - g(n)\\
			f(n) &= (a - 1)g(n)
		\end{align}
		and because $f(n) > 0$, $0 < (a-1) \leq 1$. Using multiplication and our inequality, we have
		\begin{align}
			0 < f(n) \leq g(n) \leq a \cdot g(n)\\
			0 < (a-1)g(n) \leq g(n) \leq a \cdot g(n)
		\end{align}
		for all $n \geq n_0$.
		Notice that $(6)$ is in the form that \ref{thetadef} requires. In addition, we have shown that $(a-1)$ and $a$ are positive constants and that the inequality holds for all $n \geq n_0$. If we substitute the original equations in question back into $(6)$, we get
		\[
			0 < f(n) \leq \text{max}(f(n) + g(n)) \leq f(n) + g(n)
		\]
		remembering that $f(n) \leq g(n)$. Thus, we have shown that \ref{thetadef} holds for the statement \[\text{max}(f(n), g(n)) = \Theta(f(n) + g(n))\]
	\end{proof}
	
	\section{Problem 3-1, parts a, b, and c}
	\begin{definition}
		\label{odef}
		$O(g(n)) = \{f(n) :$ there exist positive constants $c$ and $n_0$ such that $0 \leq f(n) \leq cg(n)$ for all $n \geq n_0\}$.
	\end{definition}
	\begin{definition}
		\label{omegadef}
		$\Omega(g(n)) = \{f(n) :$ there exist positive constants $c$ and $n_0$ such that $0 \leq cg(n) \leq f(n)$ for all $n \geq n_0 \}$.
	\end{definition}
	\begin{equation}
		p(n) = a_dn^d + a_{d-1}n^{d-1} +...+ a_0
	\end{equation}
	\begin{theorem}
		\label{limittheorem}
		\begin{align}
			\frac{p(n)}{n^d} &= \frac{a_dn^d + a_{d-1}n^{d-1} +...+ a_0}{n^d}\\
			&= a_d + \frac{a_{d-1}}{n} +...+ \frac{a_0}{n^d}\\
			\lim_{n \rightarrow \infty} \frac{p(n)}{n^d} &= a_d + 0 + ... + 0\\
			\lim_{n \rightarrow \infty} p(n) &= a_dn^d
		\end{align}
	Thus, for sufficiently large $n$, the growth of $p(n)$ is dominated by its highest power term $n^d$.
	\end{theorem}
	\subsection*{3.a)}
	\begin{proof}
		Let $f(n) = p(n)$ and $g(n) = n^k$. Using the approximation derived in Theorem \ref{limittheorem}, $f(n) \approx a_dn^d$ for all $n \geq n_0$ where $n_0$ is sufficiently large. In addition, it is trivial that for all $k \geq d$, $xn^k \geq xn^d$ for all $n \geq 1$. Finally, as defined, $a_d > 0$ and so $a_dn^d > 0$ for all $n \geq n_0$. Thus,
		\[
			0 \leq a_dn^d \leq cn^k
		\]
		for all $k \geq d$ and any $c \geq a_d$ which satisfies the Big-Oh definition.
	\end{proof}
	
	\subsection*{3.b)}
	\begin{proof}
		Let $f(n) = p(n)$ and $g(n) = n^k$. Using the approximation derived in Theorem \ref{limittheorem}, $f(n) \approx a_dn^d$ for all $n \geq n_0$ where $n_0$ is sufficiently large. In addition, it is trivial that for all $k \leq d$, $xn^k \leq xn^d$ for all $n \geq 1$. Finally, as defined, $a_d > 0$ and so $a_dn^d > 0$ for all $n \geq n_0$. Thus,
		\[
			0 \leq cn^k \leq a_dn^d
		\]
		for all $k \leq d$ and any $c \leq a_d$ which satisfies the Big-Omega definition.
	\end{proof}
	
	\subsection*{3.c)}
	\begin{proof}
		Let $f(n) = p(n)$ and $g(n) = n^k$. Using the approximation derived in Theorem \ref{limittheorem}, $f(n) \approx a_dn^d$ for all $n \geq n_0$ where $n_0$ is sufficiently large. In addition, since $k = d$, $n^k = n^d$. Finally, as defined, $a_d > 0$ and so $a_dn^d > 0$ for all $n \geq n_0$. To fulfill the Big-Theta definition, we must find a $c_1$ and $c_2$ such that
		\[
			0 \leq c_1n^d \leq a_dn^d \leq c_2n^d
		\]
		If we divide the entire inequality by $a_dn^d$, we get
		\[
			0 \leq c_1 \leq a_d \leq c_2
		\]
		Thus, for any $0 \leq c_1 \leq a_d$ and $c_2 \geq a_d$, the Big-Theta definition is satisfied.
	\end{proof}
	

\end{document}