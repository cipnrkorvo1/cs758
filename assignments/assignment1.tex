\documentclass[]{article}

\usepackage[a4paper, total={6in, 8in}]{geometry}
\usepackage{amsmath, amsthm, setspace}

\theoremstyle{definition}
\newtheorem{definition}{Definition}[section]

\begin{document}
	
	\begin{center}
	\Large\bfseries Assignment 1
	\end{center}
	\begin{center}
		CS 758: Algorithms\\
		Grey Cole\\
		January 29, 2026
	\end{center}
	
	\section{Exercise 2.1–3 in CLRS}
	
	\begin{verbatim}
		for i from 0 until n:
		    if A[i] == v then return i
		return NIL
	\end{verbatim}
	
	\noindent \textbf{Invariant:} At the start of each loop at index $i$, value $v$ does not appear in the subarray $A[0..i-1]$.\\\\
	
	\noindent\textbf{1 \; Initialization.} Before the first iteration $(i = 0)$, $A[0..-1]$ is empty, so $v$ cannot be contained in it.\\
	\textbf{2 \; Induction.} If $A[i] = v$ the algorithm returns $i$ so the invariant no longer needs to hold. \\\noindent If $A[i] \neq v$ then $A[0..i-1] \cup A[i]$ will not contain $v$ so the invariant holds.\\
	\textbf{3 \; Termination.} If the loop finishes without returning (when $i = n$), the invariant tells us that $v$ cannot be contained within the input array $A$ (assuming $A = A[0..n-1]$), so the algorithm returns NIL instead.\\
	
	\section{Exercise 3.1–1 in CLRS}
	
	\begin{definition}[$\Theta$-notation]
		\label{thetadef}
		For a given function $g(n)$, we denote by $\Theta(g(n))$ the \textit{set of functions}:\\
		$\Theta(g(n)) = \{ f(n) :$ there exist positive constants $c_1$, $c_2$, and $n_0$ such that $0 \leq c_1g(n) \leq f(n) \leq c_2g(n)$ for all $n \geq n_0\}$.
	\end{definition}
	
	\begin{definition}[Asymptotically nonnegative]
		\label{nonneg}
		There is a value $n_0$ such that $f(n) > 0$ for all $n \geq n_0$. 
	\end{definition}
	
	\begin{proof}
		Let $F(n) = \text{max}(f(n), g(n))$ and $G(n) = (f(n) + g(n))$. To prove that $\Theta(G(n)) = F(n)$, we must show that Definition \ref{thetadef} holds.\\\\
		
		Choose $c_1 = 1, c_2 = 10, n_0 = 3$.
	\end{proof}
	

\end{document}